
\documentclass[11pt,a4paper]{report}
% diz o tipo de documento que vamos escrever: artigo, relatório, carta, etc...(relatório no nosso caso)
% [11pt,a4paper] Define o tamanho da letra no documento, sendo que caso não seja expecificado assume tamanho 10 por default.
% a4paper - Define o tamanho do papel.

\usepackage[portuges]{babel}
% babel - Todo o texto gerado é traduzido para Português de forma automática. “chapter” irá passar a “capítulo”... 
% português - específica para o Português.

\usepackage{graphicx} 
% permite a inclusão de graficos, tabelas, figuras etc...

\usepackage{url} 
% serve para a utilização do comando \url{}

\usepackage[utf8]{inputenc} 
% Define a modelação utilizada, texto fonte (input)- usual "utf8" ou "latin1

\usepackage{enumerate} 
% permite nas listas enumeradas, a escolha de se os iems ao invés de numeração normal sao marcados com letras ou s-romanos.

\usepackage{color} 
% Serve para escrever usando cores

\usepackage{multirow} 
% serve para tabelas com multilinhas

\usepackage{array} 
% serve para a formatação especial de tabelas em forma de array

\usepackage[pdftex]{hyperref} 
% serve para transformar as referências internas do documento em hiper-ligações.

\usepackage{listings}
% serve para incluir blocos de código

\definecolor{saddlebrown}{rgb}{0.55, 0.27, 0.07} 
% serve para definir uma nova cor com o nome escolhido  neste caso "saddlebrown".

\definecolor{comentario_cod}{rgb}{0,0.4,0}
\definecolor{nlinha_cod}{rgb}{0.3,0.3,0.3}
\definecolor{azul_cod}{rgb}{0,0,0.7}
\definecolor{fundo_cod}{rgb}{0.90,0.90,0.90}
\definecolor{importfuncao_cod}{rgb}{0.7,0,0.5}

\lstdefinestyle{mystyle}{
	backgroundcolor=\color{fundo_cod},   
	commentstyle=\color{comentario_cod},
	keywordstyle=\color{importfuncao_cod},
	numberstyle=\tiny\color{nlinha_cod},
	stringstyle=\color{azul_cod},
	basicstyle=\ttfamily\footnotesize,
	breakatwhitespace=false,         
	breaklines=true,                 
	captionpos=b,                    
	keepspaces=true,                 
	numbers=left,                    
	numbersep=5pt,                  
	showspaces=false,                
	showstringspaces=false,
	showtabs=false,                  
	tabsize=2
}

\lstset{style=mystyle}

\usepackage{xspace} 
% deteta se a posteriormente à palavra tem uma palavra, dando espaço neste caso, ou um sinal de pontuaçao, se o caso for este nao dá espaço

\parindent=0pt 
% define o espaço a deixar para indentar da primeira linha após um parágrafo

\parskip=2pt 
% define o espaçamento entre o parágrafo e o texto anterior

\setlength{\oddsidemargin}{-1cm} 
% define o espaço entre a margem e o texto

\setlength{\textwidth}{18cm} 
% Define o comprimento do texto na pagina

\setlength{\headsep}{-1cm} 
% define o espaço entre o texto e o cabeçalho

\setlength{\textheight}{23cm} 
% define a altura do texto na pagina

% comando '\def' usado para definir abreviaturas (macros)
% o 1º argumento é o nome do novo comando, já o 2º entre chavetas é o texto original, para que expande
\def\darius{\textsf{Darius}\xspace}
\def\antlr{\texttt{AnTLR}\xspace}
\def\pe{\emph{Publicação Eletrónica}\xspace}
\def\titulo#1{\section{#1}}    
% no corpo do documento usa-se na forma '\titulo{MEU TITULO}'
\def\super#1{{\em Supervisor: #1}\\ }
\def\area#1{{\em \'{A}rea: #1}\\[0.2cm]}
\def\resumo{\underline{Resumo}:\\ }

\title{Processamento de Linguagens e Compiladores (3º Ano)\\
       \textbf{Trabalho Prático 2}\\ Relatório de Desenvolvimento
       }
       
\author{Alexandra Calafate\\ (a100060) \and Gonçalo Magalhães\\ (a100084)
         \and João Mirra\\ (a100083)
       }
       
\date{05/01/2025}

\begin{document} % corpo do documento
\maketitle % apresentar titulo, autor e data

\begin{abstract}
No contexto da disciplina de Processamento de Linguagens e Compiladores, o segundo projeto prático consiste em criar uma linguagem imperativa personalizada e desenvolver um compilador utilizando os módulos de gramáticas tradutoras disponíveis em Python. Este trabalho inclui a definição de uma gramática que converte a linguagem imperativa em código assembly, utilizando ferramentas como o lex e o yacc do Python. Neste relatório, descrevemos detalhadamente as escolhas realizadas, as produções implementadas na gramática e as etapas seguidas no desenvolvimento do compilador, de forma clara e objetiva.
\end{abstract}

\tableofcontents

\chapter{Introdução} \label{chap:intro} %referência cruzada
No âmbito da disciplina de Processamento de Linguagens e Compiladores, o professor Pedro Rangel Henriques propôs-nos um projeto de grupo com objetivos fundamentais. Estes incluem o desenvolvimento de competências na escrita eficiente de gramáticas, a capacitação para construir um processador de linguagens com base numa gramática tradutora e, ainda, a criação de um compilador capaz de gerar código para uma máquina de stack virtual.
\\
\\
A linguagem a ser usada neste projeto, será uma linguagem imperativa simples com regras acordadas pelos elementos do grupo. 
\\
Com base na gramática independente do contexto (GIC) que definimos, o compilador desenvolvido para a nossa linguagem terá de gerar pseudo-código Assembly para uma VM.
\\
\\
Neste documento apresentamos uma possivel solução para cada um dos problemas propostos, com recurso aos módulos 'Yacc/Lex' do 'PLY/Python'.

\newpage

\section*{Estrutura do Relatório}
O relatório está organizado da seguinte forma:\\
Iniciamos com uma breve introdução, no capítulo 1, onde é descrito o objetivo do trabalho a desenvolver.\\
No capítulo 2, é apresentado o enunciado do problema proposto.\\
No capítulo 3, descrevemos a estrutura do nosso trabalho.\\
No capítulo 4, demonstramos o funcionamento de vários testes realizados pelo grupo.\\
Por fim, no último capítulo, apresentamos a conclusão e uma análise final do trabalho realizado.

\newpage

\chapter{Problema Proposto}

Pretende-se que comece por definir uma linguagem de programação imperativa simples, a seu gosto.
Apenas deve ter em consideração que essa linguagem terá de permitir:
\begin{itemize}
	\item declarar variáveis atómicas do tipo inteiro, com os quais se podem realizar as habituais operações aritméticas, relacionais e lógicas.
	\item efetuar instruções algorítmicas básicas como a atribuição do valor de expressões numéricas a variáveis.
	\item ler do standard input e escrever no standard output.
    \item efetuar instruções de seleção para o controlo do fluxo de execução.
    \item efetuar instruções de repetição(cíclicas) para o controlo de fluxo de execução, permitindo o seu aninhamento.\\
    \underline{Note} que deve implementar pelo menos o ciclo \textbf{while-do},\textbf{repeat-until} ou \textbf{for-do}.
\end{itemize}
Adicionalmente deve ainda suportar,à sua escolha, uma das duas funcionalidades seguintes:
\begin{itemize}
    \item declarar e manusear variáveis estruturadas do tipo array( a 1 ou 2 dimensões) de inteiros, em relação aos quais é apenas permitida a operação de indexação(índice inteiro).
    \item definir e invocar subprogramas sem parâmetros mas que possam retornar um resultado do tipo inteiro.
\end{itemize}

\newpage

\chapter{Elaboração da Solução}
\section{Organização e estrutura}
O nosso trabalho pode ser divido em \textbf{4} partes:
\begin{itemize}
\item Cosntrução da \textbf{GIC} que define a estrutura sintática da liguaguem.
\item Construção do analizador léxico, \textbf{lexer} .
\item Construção do analizador sintático, \textbf{parser} .
\item Conversão das instruções para código \textbf{Assembly} da  VM.    
\end{itemize}
Todas as funcionalidades descritas neste capítulo podem ser encontradas no anexo A do documento. \\ \\

\section{GIC}
A nossa linguagem será gerada pela seguinte Grámatica Independente de Contexto (GIC):
\begin{verbatim}

## GIC que criamos

Programa : Decls
         | Atrib
         | Decls Corpo
         | Atrib Corpo
         | Corpo

Corpo : Codigo
    
Codigo : Proc Codigo
       | Proc
       | Decls Codigo
       | Decls
       | Atrib Codigo
       | Atrib
       | COMENTARIO Codigo
       | COMENTARIO

Decls : Decl PONTOVIRGULA
      | Decl VIRGULA Decls

Proc : IF
     | WHILE
     | PRINTAR

Decl : INT ID
     | MATRIZ INT NUM NUM VIRGULA ID
     | LISTA INT NUM VIRGULA ID PONTOVIRGULA

Atrib : INT ID IGUAL expr PONTOVIRGULA
      | ID IGUAL expr PONTOVIRGULA
      | ALTERA ID IGUAL expr PONTOVIRGULA
      | ALTERA ID LPAREN_RETO expr RPAREN_RETO IGUAL expr PONTOVIRGULA                                  -- Lista
      | ALTERA ID LPAREN_RETO expr RPAREN_RETO LPAREN_RETO expr RPAREN_RETO IGUAL expr PONTOVIRGULA     -- Matriz
      | ALTERA ID LPAREN_RETO expr RPAREN_RETO IGUAL LISTA                                              -- add Lista em Matriz

expr : exprArit
     | exprRel
     | NUM
     | ID
     | INPUT
     | ID LPAREN_RETO expr RPAREN_RETO
     | ID LPAREN_RETO expr RPAREN_RETO LPAREN_RETO expr RPAREN_RETO

exprArit : exprArit PLUS term
         | exprArit MINUS term
         | term

term : term TIMES factor
     | term DIVIDE factor
     | factor

factor : NUM
       | ID

exprRel : expr IGUALIGUAL expr
        | expr DIFERENTE expr
        | expr MENOR expr
        | expr MENOROUIGUAL expr
        | expr MAIOR expr
        | expr MAIOROUIGUAL expr
        | expr EE expr
        | expr OU expr

IF : SE LPAREN exprRel RPAREN ENTAO Codigo FIM_COND PONTOVIRGULA
   | SE LPAREN exprRel RPAREN ENTAO Codigo SENAO Codigo FIM_COND PONTOVIRGULA

WHILE : ENQUANTO LPAREN exprRel RPAREN FAZ Codigo FIM_ENQUANTO PONTOVIRGULA

PRINTAR : IMPRIMIR LPAREN FRASE RPAREN PONTOVIRGULA
        | IMPRIMIR LPAREN ID RPAREN PONTOVIRGULA

COMENTARIO : COMENTADO
          
\end{verbatim}

\newpage

\section{Lexer}
Através de expressões regulares, o analisador léxico \textbf{lexer} tem a função de identificar os símbolos terminais (\textit{tokens}) da nossa linguagem. Para a sua implementação, utilizámos o módulo 'Lex' do 'PLY/Python'.\ \
Os \textit{tokens} e as respetivas expressões regulares da nossa linguagem são os seguintes:

\begin{verbatim}

tokens = [
    'ID',
    'NUM',
    'INT',

    'PLUS',
    'MINUS',
    'TIMES',
    'DIVIDE',

    'MAIOR',
    'MAIOROUIGUAL',
    'MENOR',
    'MENOROUIGUAL',
    'IGUALIGUAL',
    'DIFERENTE',
    'EE',
    'OU',

    'LPAREN',
    'RPAREN',
    'LPAREN_RETO',
    'RPAREN_RETO',

    'SE',
    'ENTAO',
    'SENAO',
    'FIM_COND',

    'ENQUANTO',
    'FAZ',
    'FIM_ENQUANTO',

    'INPUT',
    'IMPRIMIR',
    'VIRGULA',
    'PONTOVIRGULA',
    'IGUAL',

    'COMENTADO',

    'FRASE',

    'ALTERA',

    'LISTA',
    'MATRIZ'
]

# Regras dos tokens
t_INT = r'int'
t_ID = r'\w+'

t_PLUS = r'\+'
t_MINUS = r'-'
t_TIMES = r'\*'
t_DIVIDE = r'/'

t_MAIOR = r'>'
t_MAIOROUIGUAL = r'>='
t_MENOR = r'<'
t_MENOROUIGUAL = '<='
t_IGUALIGUAL = '=='
t_DIFERENTE = '!='
t_EE = '\\/'
t_OU = '\\/'

t_LPAREN = r'\('
t_RPAREN = r'\)'
t_LPAREN_RETO = r'\['
t_RPAREN_RETO = r'\]'

t_SE = r'ifse'
t_ENTAO = r'entao'
t_SENAO = r'senao'
t_FIM_COND = r'fim_cond'

t_ENQUANTO = r'enquanto'
t_FAZ = r'faz'
t_FIM_ENQUANTO = r'fim_enquanto'

t_INPUT = r'input'
t_IMPRIMIR = r'imprimir'
t_VIRGULA = r','
t_PONTOVIRGULA = r';'
t_IGUAL = r'='

t_COMENTADO = r'%[^\n]*'

t_FRASE = r'"[^"]+"'

t_ALTERA = r'altera'

t_LISTA = r'lista'
t_MATRIZ = r'matriz'

\end{verbatim}

A implementação do analisador léxico pode ser encontrada no anexo ao documento. 
\\ \\

\section{Parser e geração do código Assembly da VM}

O \textbf{parser}, ou analisador sintático, tem a função de assegurar que o código escrito na nossa linguagem está corretamente estruturado de acordo com as regras gramaticais predefinidas. Em outras palavras, ele verifica se o código segue a sintaxe estabelecida. \ \

Caso não haja erros sintáticos, o \textbf{parser} realiza a transformação do código na nossa linguagem para o código \textbf{Assembly} da máquina virtual. Se forem encontrados erros ou anomalias, uma mensagem de erro sintático será exibida ao utilizador. \ \

A implementação deste analisador sintático está completa num dos anexos do nosso relatório, conforme solicitado pelo docente.

\subsection{Notas sobre declaração de variáveis}

Ao gerar o código para fazer a declaração de uma variável sem valor fazemos:
\begin{lstlisting}
PUSHI 0    
\end{lstlisting}
Sendo que variável tem valor default igual a 0.\\

Por outro lado, para declará-la com valor fazemos:
\begin{lstlisting}
PUSHI <valor>
STOREG <endereco>
\end{lstlisting}
No caso duma lista de valores fazemos:
\begin{lstlisting}
PUSHN <tamanho>
\end{lstlisting}
Para Inicializar todos os valores do array como 0.\\ \\
Seja uma lista de tamanho 2 temos por exemplo:
\begin{lstlisting}
lista int 2, nomelista;
\end{lstlisting}
Para atribuir valores à lista fazemos:
\begin{lstlisting}
PUSHN <tamanho>
PUSHGP
PUSHI 0
PUSHI <valor1>
STOREN
PUSHGP
PUSHI 1
PUSHI <valor2>
STOREN
\end{lstlisting}

Para declarar uma matriz procedemos a: 
\begin{lstlisting}
matriz int <tamanho1> <tamanho2>, <nomematriz>;   
\end{lstlisting}
Caso a matriz m seja com formato 2x2, é gerado:
\begin{lstlisting}
PISHN 4
\end{lstlisting}
Os valores são inicializados a 0. \\

Se pretendermos modificar os valores duma matriz temos de proceder da seguinte forma:
\begin{enumerate}
\item Alteramos uma especifica posição:
\begin{lstlisting}
altera <nomeMatriz>[<pos1>][<pos2>] = <valor>    
\end{lstlisting}
\end{enumerate}

Tendo uma matriz 2x2 como exemplo, fazendo a alteração de valores temos:
\begin{lstlisting}
altera m [0][1] com <valor>
\end{lstlisting}
Gera-se o seguinte:
\begin{lstlisting}
PUSHN 4
PUSHGP
PUSHI 0
PADD
PUSHI 0
PUSHI <valor>
STOREN
\end{lstlisting}

\chapter{Demonstração do Funcionamento}
\section{Geração e  execução de código Assembly}
Para escrever instruç˜oes de acordo com as regras gramaticais da linguagem, o utilizador tem 2 opções de execução:
\begin{enumerate}
\item Caso esteja na mesma pasta dos restantes ficheiros:
\begin{verbatim} 
>> python yacc.py <ficheiro de input> 
\end{verbatim}
Por exemplo:
\begin{verbatim} 
>> python yacc.py testeA.plc
\end{verbatim} 
\item Caso esteja numa outra pasta:
\begin{verbatim} 
>> python yacc.py <.\pasta\ficheiro de input>
\end{verbatim} 

Por exemplo:
\begin{verbatim} 
>> python yacc.py .\testes\testeA.plc
\end{verbatim} 
\end{enumerate}
\section{Teste 1}
\underline{Ficheiro de input:} 'testeA.plc'
\subsection{Conteúdo do ficheiro}

\begin{lstlisting}[language=python]
int x,int y, int resultado;

x = 4;
y = 1;

resultado = x+y;

imprimir(resultado);
\end{lstlisting}

\subsection{Código assembly gerado}
\begin{lstlisting}[language=Assembler]
PUSHI 0
PUSHI 0
PUSHI 0
START
PUSHI 4
STOREG 3
PUSHI 1
STOREG 4
PUSHG 3
PUSHG 4
ADD
STOREG 5
PUSHG 5
WRITEI
STOP
\end{lstlisting}

\subsection{Execução da VM com o código gerado}
\begin{lstlisting}[language=python]
5
\end{lstlisting}

\section{Teste 2}
\underline{Ficheiro de input:} 'testeB.plc'.

\subsection{Conteúdo do ficheiro}

\begin{lstlisting}[language=python]
int x;

x = 4;

enquanto (x > 0) faz
    altera x = x-1;
fim_enquanto;

imprimir(x);
\end{lstlisting}

\subsection{Código assembly gerado}
\begin{lstlisting}[language=Assembler]
PUSHI 0
START
PUSHI 4
STOREG 1
l0c: NOP
PUSHG 1
PUSHI 0
SUP
JZ l0f
PUSHG 1
PUSHI 1
SUB
STOREG 1
JUMP l0c
l0f: NOP
PUSHG 1
WRITEI
STOP
\end{lstlisting}

\subsection{Execução da VM com o código gerado}
\begin{lstlisting}[language=python]
0
\end{lstlisting}

\section{Teste 3}

\underline{Ficheiro de input:} 'testeC.plc'.

\subsection{Conteúdo do ficheiro}

\begin{lstlisting}[language=python]
int x;

x = 4;

ifse (x>1) entao 
    imprimir("maior");
fim_cond;
\end{lstlisting}

\subsection{Código assembly gerado}
\begin{lstlisting}[languague=Assembler]
PUSHI 0
START
PUSHI 4
STOREG 1
PUSHG 1
PUSHI 1
SUP
JZ l0
PUSHS "maior"
WRITES
l0: NOP
STOP
\end{lstlisting}

\subsection{Execução da VM com o código gerado}
\begin{lstlisting}[language=python]
maior
\end{lstlisting}

\section{Teste 4}

\underline{Ficheiro de input:} 'testeD.plc'.

\subsection{Conteúdo do ficheiro}

\begin{lstlisting}[language=python]
int x;

x = 4;

ifse (x>5) entao 
    imprimir("maior");
senao
    imprimir("menor");
fim_cond;
\end{lstlisting}

\subsection{Código assembly gerado}
\begin{lstlisting}[languague=Assembler]
PUSHI 0
START
PUSHI 4
STOREG 1
PUSHG 1
PUSHI 5
SUP
JZ l0
PUSHS "maior"
WRITES
JUMP l0f
l0: NOP
PUSHS "menor"
WRITES
l0f: NOP
STOP
\end{lstlisting}

\subsection{Execução da VM com o código gerado}
\begin{lstlisting}[language=python]
menor
\end{lstlisting}

\section{Teste 5}

\underline{Ficheiro de input:} 'testeE.plc'.

\subsection{Conteúdo do ficheiro}

\begin{lstlisting}[language=python]
int x;

lista int 2, arr;

altera arr[0] = 1;

x = arr[0];

imprimir(x);
\end{lstlisting}

\subsection{Código assembly gerado}
\begin{lstlisting}[languague=Assembler]
PUSHI 0
START
PUSHN 2
PUSHGP
PUSHI 1
PADD
PUSHI 0
PUSHI 1
STOREN
PUSHGP
PUSHI 1
PADD
PUSHI 0
LOADN
STOREG 3
PUSHG 3
WRITEI
STOP
\end{lstlisting}

\subsection{Execução da VM com o código gerado}
\begin{lstlisting}[language=python]
1
\end{lstlisting}

\section{Teste 6}

\underline{Ficheiro de input:} 'testeF.plo'.

\subsection{Conteúdo do ficheiro}

\begin{lstlisting}[language=python]
int x;
int y;

x = 2*4 + 5;
y = 5 + 4*2;    

imprimir(x);
imprimir("\n");
imprimir(y);
\end{lstlisting}

\subsection{Código assembly gerado}
\begin{lstlisting}[languague=Assembler]
PUSHI 0
START
PUSHI 0
PUSHI 2
PUSHI 4
MUL
PUSHI 5
ADD
STOREG 2
NOP
PUSHI 5
PUSHI 4
PUSHI 2
MUL
ADD
STOREG 3
PUSHG 2
WRITEI
PUSHS "\n"
WRITES
PUSHG 3
WRITEI
STOP
\end{lstlisting}

\subsection{Execução da VM com o código gerado}
\begin{lstlisting}[language=python]
13
13
\end{lstlisting}

\chapter{Conclusão}

Ao longo deste projeto, procuramos constantemente aplicar os conhecimentos adquiridos nas aulas (teóricas e práticas), o que nos permitiu aprofundar e solidificar de maneira mais eficaz os conceitos abordados nesta Unidade Curricular.

Consideramos que, de forma geral, conseguimos atingir os objetivos preestabelecidos, o que nos proporcionou uma maior competência na elaboração de gramáticas e no desenvolvimento de compiladores de linguagens. Embora reconheçamos que poderíamos ter implementado uma gramática mais eficiente, acreditamos que a gramática criada atende aos requisitos do trabalho e resolve os problemas da linguagem em questão. Além disso, esta iniciativa contribuiu de forma significativa para ampliar o nosso entendimento sobre a máquina virtual e aprimorar as nossas habilidades na linguagem Assembly.

Em resumo, todo o empenho dedicado à realização deste projeto mostrou-se extremamente benéfico, consolidando as nossas bases e proporcionando uma maior facilidade na abordagem de temas específicos da Unidade Curricular, que podem ser relevantes em uma futura trajetória profissional.

\appendix % apendice 
\chapter{Código do Programa}
\section{Apêndice}
\textbf{Ficheiro lex.py}
\begin{scriptsize}
\begin{lstlisting}[language=python]
# lex.py

import ply.lex as lex # Import do ply.lex

# Definição dos tokens
tokens = [
    'ID',
    'NUM',
    'INT',

    'PLUS',
    'MINUS',
    'TIMES',
    'DIVIDE',

    'MAIOR',
    'MAIOROUIGUAL',
    'MENOR',
    'MENOROUIGUAL',
    'IGUALIGUAL',
    'DIFERENTE',
    'EE',
    'OU',

    'LPAREN',
    'RPAREN',
    'LPAREN_RETO',
    'RPAREN_RETO',

    'SE',
    'ENTAO',
    'SENAO',
    'FIM_COND',

    'ENQUANTO',
    'FAZ',
    'FIM_ENQUANTO',

    'INPUT',
    'IMPRIMIR',
    'VIRGULA',
    'PONTOVIRGULA',
    'IGUAL',

    'COMENTADO',

    'FRASE',

    'ALTERA',

    'LISTA',
    'MATRIZ'
]

# Regras dos tokens
t_INT = r'int'
t_ID = r'\w+'

t_PLUS = r'\+'
t_MINUS = r'-'
t_TIMES = r'\*'
t_DIVIDE = r'/'

t_MAIOR = r'>'
t_MAIOROUIGUAL = r'>='
t_MENOR = r'<'
t_MENOROUIGUAL = '<='
t_IGUALIGUAL = '=='
t_DIFERENTE = '!='
t_EE = '\\/'
t_OU = '\\/'

t_LPAREN = r'\('
t_RPAREN = r'\)'
t_LPAREN_RETO = r'\['
t_RPAREN_RETO = r'\]'

t_SE = r'ifse'
t_ENTAO = r'entao'
t_SENAO = r'senao'
t_FIM_COND = r'fim_cond'

t_ENQUANTO = r'enquanto'
t_FAZ = r'faz'
t_FIM_ENQUANTO = r'fim_enquanto'

t_INPUT = r'input'
t_IMPRIMIR = r'imprimir'
t_VIRGULA = r','
t_PONTOVIRGULA = r';'
t_IGUAL = r'='

t_COMENTADO = r'%[^\n]*'

t_FRASE = r'"[^"]+"'

t_ALTERA = r'altera'

t_LISTA = r'lista'
t_MATRIZ = r'matriz'

def t_NUM(t):
    r'\d+'
    t.value = int(t.value)
    return t

# Ignorar espacos em branco e tabulacoes \t
t_ignore = ' \t'

# Tratamento do \n
def t_newline(t):
    r'\n+'
    t.lexer.lineno += len(t.value)

# Erro 
def t_error(t):
    print(f"Erro no caracter: '{t.value[0]}'")
    t.lexer.skip(1)

# Analisador léxico
lexer = lex.lex()


\end{lstlisting}
\end{scriptsize}

\chapter{Código do Programa}
\section{Apêndice}
\textbf{Ficheiro yacc.py}
\begin{scriptsize} 
\begin{lstlisting}[language=python]

import ply.yacc as yacc
from lex import *   # Import do nosso ficheiro criado 

# Imports para manipulação de files necessários
import sys
import os

# Programa apenas com declarações ou atribuições
def p_Programa_Empty(p):
    '''
    Programa : Decls
             | Atrib
    '''
    parser.assembly = f'{p[1]}'

# Programa com corpo pós declarações e atribuições
def p_Programa(p):
    '''
    Programa : Decls Corpo
             | Atrib Corpo
    '''
    parser.assembly = f'{p[1]}START\n{p[2]}STOP\n'

# Programa -> Corpo
def p_Programa_Corpo(p):
    '''
    Programa : Corpo
    '''
    parser.assembly = f"START\n{p[1]}STOP\n"

# Corpo -> Codigo
def p_Corpo(p):
    '''
    Corpo : Codigo
    '''
    p[0] = f"{p[1]}"

# Codigo recursivamente
def p_Codigo_Rec(p):
    '''
    Codigo : Proc Codigo
           | Decls Codigo
           | Atrib Codigo
           | COMENTARIO Codigo
    '''
    p[0] = f"{p[1]}{p[2]}"

# Codigo pode ser um processo, atribuicao, declaracao ou um comentario
def p_Codigo(p):
    '''
    Codigo : Proc
           | Atrib
           | Decls
           | COMENTARIO
    '''
    p[0] = f"{p[1]}"

# declaracao unica terminada em ponto e virgula
def p_Decls(p):
    "Decls : Decl PONTOVIRGULA"
    p[0] = f'{p[1]}'

# declaração recursiva separada por virgulas
def p_DeclsRec(p):
    "Decls : Decl VIRGULA Decls"
    p[0] = f'{p[1]}{p[3]}'

# expr pode ser arit ou relativa
def p_expr_arit(p):
    '''
    expr : exprArit
         | exprRel
    '''
    p[0] = p[1]

# procedimento pode ser um if,while ou um print
def p_Proc(p):
    '''
    Proc : IF
         | WHILE
         | PRINTAR
    '''
    p[0] = p[1]


# Declaração de uma variavel sem valor
def p_Decl(p):
    "Decl : INT ID"
    varName = p[2]

    # é verificado no dicionario de variaveis se ela ja existe
    if varName not in parser.variaveis:
        parser.variaveis[varName] = (parser.stackPointer, None)
        p[0] = "PUSHI 0\n"                  # Atribuido valor 0 por definição
        parser.stackPointer += 1
    else:
        parser.exito = False
        parser.error = f"Variável com o nome {varName} já existe"
    parser.linhaDeCodigo +=1

# Declaração de uma variável com atribuição de um valor
def p_Atrib_expr(p):
    "Atrib : INT ID IGUAL expr PONTOVIRGULA"
    varName = p[2]

    # é verificado no dicionario de variaveis se ela ja existe
    if varName not in parser.variaveis:
        value = p[4]
        parser.variaveis[varName] = (parser.stackPointer, None)
        p[0] = f"{value}STOREG {parser.stackPointer}\n"
        parser.stackPointer += 1
    else:
        parser.exito = False
        parser.error = f"Variável com o nome {varName} já existe"
    parser.linhaDeCodigo +=1

# Atribuição de uma variavel ja declarada
def p_Atrib_sem_decl(p):
    "Atrib : ID IGUAL expr PONTOVIRGULA"
    varName = p[1]

    # Verifica se a tal variavel já existe mesmo
    if varName in parser.variaveis:
        value = p[3]
        parser.variaveis[varName] = (parser.stackPointer, None)
        p[0] = f"{value}STOREG {parser.stackPointer}\n"
        parser.stackPointer += 1
    else:
        parser.exito = False
        parser.error = f"Variável com o nome {varName} não existe"
    parser.linhaDeCodigo +=1


# Altera valor de um variável
def p_alterna_var(p):
    "Atrib : ALTERA ID IGUAL expr PONTOVIRGULA"
    varName = p[2]

    # verifica se a variavel ja existe mesmo
    if varName in parser.variaveis:
        p[0] = f"{p[4]}STOREG {parser.variaveis[varName][0]}\n"
    parser.linhaDeCodigo +=1

# expr pode ser um simples valor
def p_expr(p):
    "expr : NUM"
    p[0] = f"PUSHI {int(p[1])}\n"

# expr pode ser uma variavel
def p_expr_var(p):
    "expr : ID"
    varName = p[1]
    if varName in parser.variaveis:
        p[0] = f"PUSHG {parser.variaveis[varName][0]}\n"

# expr pode receber input
def p_expr_input(p):
    "expr : INPUT"
    p[0] = f"READ\nATOI\n"


# Declara lista com tamanho INT
def p_DeclLista_Size(p):
    '''Decl : LISTA INT NUM VIRGULA ID''' # lista int 5, arr;
    listName = p[5]
    size = int(p[3])

    # verifica se esse nome já não esta a ser usado noutra variavel
    if listName not in parser.variaveis:
        # tamanho obrigatoriamente maior que 0
        if size > 0:
            parser.variaveis[listName] = (parser.stackPointer, size)
            p[0] = f"PUSHN {size}\n"
            parser.stackPointer += size
        else:
            parser.error = f"Impossível declarar um array de tamanho {size}"
            parser.exito = False
    else:
        parser.error = (
            f"Variável com o nome {listName} já definida anteriormente.")
        parser.exito = False
    parser.linhaDeCodigo +=1


# Altera valor de um indice da lista
def p_AlteraLista_elem(p):
    "Atrib : ALTERA ID LPAREN_RETO expr RPAREN_RETO IGUAL expr PONTOVIRGULA" # altera arr[3] = 4;
    varName = p[2]
    
    # garante que a lista esta declarada
    if varName in parser.variaveis:
        p[0] = f"PUSHGP\nPUSHI {parser.variaveis[varName][0]}\nPADD\n{p[4]}{p[7]}STOREN\n"
    else:
        parser.error = f"Variável com o nome {varName} não definida"
        parser.exito = False
    parser.linhaDeCodigo +=1

# Função que vai buscar o valor do indice na lista
def p_AtribBusca_Lista(p):
    "expr : ID LPAREN_RETO expr RPAREN_RETO" # x = arr[4];
    varName = p[1]
    indice = p[3]
    
    # garante que a lista existe
    if varName in parser.variaveis:
        p[0] = f"PUSHGP\nPUSHI {parser.variaveis[varName][0]}\nPADD\n{indice}LOADN\n"
    else:
        parser.error = (
            f"Variável com o nome {varName} não definida anteriormente.")
        parser.exito = False
    parser.linhaDeCodigo += 1

# Declara matriz com tamanho INT INT
def p_DeclMatriz(p):
    "Decl : MATRIZ INT NUM NUM VIRGULA ID" # matriz int 3 2, mat;
    listName = p[6]
    size = int(p[3])
    size1 = int(p[4])

    # Garante que o nome escolhido já não esta a ser usado
    if listName not in parser.variaveis:
        parser.variaveis[listName] = (parser.stackPointer, size, size1)
        p[0] = f"PUSHN {size*size1}\n"
        parser.stackPointer += size*size1
    else:
        parser.error = (
            f"Variável com o nome {listName} já definida anteriormente.")
        parser.exito = False
    parser.linhaDeCodigo +=1

# Função que altera o valor de um indice da matriz por outro
def p_AtribMatriz_comExpr(p):
    "Atrib : ALTERA ID LPAREN_RETO expr RPAREN_RETO LPAREN_RETO expr RPAREN_RETO IGUAL expr PONTOVIRGULA" # altera mat[0][1] = 2;
    matName = p[2]
    indice1 = p[4]
    indice2 = p[7]
    valor = p[10]

    # Garante que a matriz existe
    if matName in parser.variaveis:
        if len(parser.variaveis[matName]) == 3:
            p[0] = f"PUSHGP\nPUSHI {parser.variaveis[matName][0]}\nPADD\n{indice1}PUSHI {parser.variaveis[matName][2]}\nMUL\nPADD\n{indice2}{valor}STOREN\n"
        else:
            parser.error = f"Operação inválida, variável {matName} não é uma matriz"
            parser.exito = False
    else:
        parser.error = f"Variável não declarada anteriormente"
        parser.exito = False
    parser.linhaDeCodigo +=1

# Função que vai buscar o valor do indice na matriz
def p_AtribBusca_Matriz(p):
    "expr : ID LPAREN_RETO expr RPAREN_RETO LPAREN_RETO expr RPAREN_RETO"                   # para usar tipo x = mat[1][2]
    varName = p[1]
    indice1 = p[3]
    indice2 = p[6]
    
    # garante que a matriz existe
    if varName in parser.variaveis:
        p[0] = f"PUSHGP\nPUSHI {parser.variaveis[varName][0]}\nPADD\n{indice1}PUSHI {parser.variaveis[varName][2]}\nMUL\nPADD\n{indice2}LOADN\n"
    else:
        parser.error = f"Variável com o nome {varName} não definida"
        parser.exito = False
    parser.linhaDeCodigo +=1

# Expressão Aritmética da soma
def p_PLUS(p):
    "exprArit : exprArit PLUS term"
    p[0] = f"{p[1]}{p[3]}ADD\n"

# Expressão Aritmética sub
def p_MINUS(p):
    "exprArit : exprArit MINUS term"
    p[0] = f"{p[1]}{p[3]}SUB\n"

# Expressao Atrimética pode ser um termo apenas
def p_expr_arit_term(p):
    '''exprArit : term'''
    p[0] = p[1]

# Expressão Aritmética da mult
def p_TIMES(p):
    "term : term TIMES factor"
    p[0] = f"{p[1]}{p[3]}MUL\n"

# Expressão Aritmética da div
def p_DIVIDE(p):
    "term : term DIVIDE factor"
    p[0] = f"{p[1]}{p[3]}DIV\n"

# term pode ser um factor
def p_term(p):
    "term : factor"
    p[0] = p[1]

# um factor pode ser um numero
def p_factor_num(p):
    '''factor : NUM'''
    p[0] = f"PUSHI {int(p[1])}\n"

# um factor tmb pode ser uma variavel
def p_factor_id(p):
    '''factor : ID'''
    varName = p[1]

    # garante que a var existe
    if varName in parser.variaveis:
        p[0] = f"PUSHG {parser.variaveis[varName][0]}\n"

# Expressão Relativa do igual
def p_IGUALIGUAL(p):
    "exprRel : expr IGUALIGUAL expr"
    p[0] = f"{p[1]}{p[3]}EQUAL\n"

# Expressão Relativa da diff (!=)
def p_DIFERENTE(p):
    "exprRel : expr DIFERENTE expr"
    p[0] = f"{p[1]}{p[3]}NOT\nEQUAL\n"

# Expressão Relativa menor
def p_MENOR(p):
    "exprRel : expr MENOR expr"
    p[0] = f"{p[1]}{p[3]}INF\n"

# Expressão Relativa menor ou igual
def p_MENOROUIGUAL(p):
    "exprRel : expr MENOROUIGUAL expr"
    p[0] = f"{p[1]}{p[3]}INFEQ\n"

# Expressão Relativa maior
def p_MAIOR(p):
    "exprRel : expr MAIOR expr"
    p[0] = f"{p[1]}{p[3]}SUP\n"

# Expressão Relativa maior ou igual
def p_MAIOROUIGUAL(p):
    "exprRel : expr MAIOROUIGUAL expr"
    p[0] = f"{p[1]}{p[3]}SUPEQ\n"

# Expressão Relativa EE (and)
def p_EE(p):
    "exprRel : expr EE expr" 
    p[0] = f"{p[1]}{p[3]}ADD\nPUSHI 2\nEQUAL\n"

# Expressão Relativa "OU" (or)
def p_OU(p):
    "exprRel : expr OU expr"
    p[0] = f"{p[1]}{p[3]}ADD\nPUSHI 1\nSUPEQ\n"

# Controlo de fluxo (if then)
def p_IF_THEN(p):
    "IF : SE LPAREN exprRel RPAREN ENTAO Codigo FIM_COND PONTOVIRGULA"
    p[0] = f"{p[3]}JZ l{parser.labels}\n{p[6]}l{parser.labels}: NOP\n"  # NOP não faz nada na vm
    parser.labels += 1
    parser.linhaDeCodigo+=1

# Controlo de fluxo (if then else)
def p_IF_THEN_ELSE(p):
    "IF : SE LPAREN exprRel RPAREN ENTAO Codigo SENAO Codigo FIM_COND PONTOVIRGULA"
    p[0] = f"{p[3]}JZ l{parser.labels}\n{p[6]}JUMP l{parser.labels}f\nl{parser.labels}: NOP\n{p[8]}l{parser.labels}f: NOP\n" # NOP não faz nada na vm
    parser.labels += 1
    parser.linhaDeCodigo+=1

# Ciclo While
def p_WHILE(p):
    "WHILE : ENQUANTO LPAREN exprRel RPAREN FAZ Codigo FIM_ENQUANTO PONTOVIRGULA"
    p[0] = f'l{parser.labels}c: NOP\n{p[3]}JZ l{parser.labels}f\n{p[6]}JUMP l{parser.labels}c\nl{parser.labels}f: NOP\n' # NOP não faz nada na vm
    parser.labels += 1
    parser.linhaDeCodigo+=1

# Print duma frase (string)
def p_PRINTAR_ID(p):
    '''PRINTAR : IMPRIMIR LPAREN FRASE RPAREN PONTOVIRGULA'''
    p[0] = f'PUSHS {p[3]}\nWRITES\n'
    parser.linhaDeCodigo+=1

# print do valor de uma variavel
def p_PRINTAR_var(p):
    '''PRINTAR : IMPRIMIR LPAREN ID RPAREN PONTOVIRGULA'''
    p[0] = f'PUSHG {parser.variaveis[p[3]][0]}\nWRITEI\n'
    parser.linhaDeCodigo+=1

# Codigo Comentado
def p_COMENTARIO(p):
    '''COMENTARIO : COMENTADO'''
    p[0] = "NOP\n" # NOP não faz nada na vm

def p_error(p):
    print(f"Erro de sintaxe na entrada: {p.value}")


##
parser = yacc.yacc()
parser.exito = True
parser.error = ""
parser.assembly = ""        # Codigo pra VM que vai sendo gerado
parser.variaveis = {}       # dicionario que armazena as variaveis declaradas
parser.stackPointer = 0     # necessario para a gestão da Stack da VM
parser.linhaDeCodigo = 0
parser.labels = 0

assembly = ""           # codigo para a VM final


if len(sys.argv) == 2:
    inputFileName = sys.argv[1]
    if inputFileName[-4:] == ".plc":            # ficheiro de entrada precisa da extensão ".plc"
        file = open(inputFileName, "r")
        content = file.read()
        parser.parse(content)
        if parser.exito:
            assembly += parser.assembly         # Atribuição do codigo para a VM
            print(parser.variaveis)
        else:
            print("Erro:")
            print(parser.error)
            print(parser.variaveis)
            sys.exit()
        file.close()
        outputFileName = "a.vm"                 # cria um file chamado "a.vm" para rodar na VM

        # Verifica se o arquivo de saída já existe e o remove antes de criar um novo
        if os.path.exists(outputFileName):
            os.remove(outputFileName)

        outputFile = open(outputFileName, "w")
        outputFile.write(assembly)              # Escrita do codigo para a VM Final
        outputFile.close()

        print("File saved successfully")

    else:
        print("Invalid file extension")

\end{lstlisting}
\end{scriptsize}

\end{document} 